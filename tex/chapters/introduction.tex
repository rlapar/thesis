\chapter*{Introduction}
\addcontentsline{toc}{chapter}{Introduction}

Public-key cryptography has gained its importance in late 70's. One of the most commonly used cryptosystem nowdays is RSA cryptosystem in many variants (768, 1024, 3072, $\cdots$). The essential demand of this cryptosystem is security. Many different implementations of RSA exist. Each implementation chooses slightly different approach to generate public and private keys to achieve right balance between security, efficiency and randomness. Some implementations are openly known, like OpenSSL, but some other like Microsoft CryptoAPI or Yubikey are not published. With each implementation different, there is a slight bias introduced in public keys generation, which leads to distinct sample space for different sources. 

These biases can be completely harmless, or they can have serious security impacts. Example of this is recently discovered flaw in Infineon chips\cite{svenda_3}, which affected more than 50\% of Estonian eID's. This could potentially lead to forging of e-signatures of the residents. The same chip was also used in Slovak eID's, albeit with less volume of affected people.

The internet contains huge amount of randomly collected dumps of RSA public keys collected from unknown sources. If one could accurately link a particular key to its source, knowing the weakness of this source, he would be able to use specific factorization method to break the public key.

Based on the previous work\cite{svenda_1}\cite{svenda_2} of the research team of CROCS lead by Petr Švenda, we would like to be able to identify a source for any key with high probability. We obtained a dataset of more than 100 million public keys from 64 different sources. We tried and compared different machine learning classifiers \textbf{TODO}

\begin{itemize}

\item provided a tool to create models faster

\end{itemize}
