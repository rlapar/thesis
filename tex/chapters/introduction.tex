\chapter*{Introduction}
\addcontentsline{toc}{chapter}{Introduction}

Public-key cryptography has gained its importance in the late 70's. One of the most commonly used cryptosystem nowdays is RSA cryptosystem in many variants (768, 1024, 3072, $\cdots$). The essential demand of this cryptosystem is security. Many different implementations of RSA exist. Each implementation chooses a slightly different approach to generate public and private keys to achieve the right balance between security, efficiency, and randomness. Some implementations are openly known, like OpenSSL, but some other like Microsoft CryptoAPI or Yubikey are not published. With each implementation being different, there is a slight bias introduced in public keys generation, which leads to distinct sample space for different sources. 

These biases can be completely harmless, or they can have serious security impacts. Example of this is a recently discovered flaw in Infineon chips\cite{svenda_2}, which affected more than 50\% of Estonian eID's. This could potentially lead to forging of e-signatures of the residents. The same chip was also used in Slovak eID's, albeit with less volume of affected people.

The internet contains a huge amount of randomly collected dumps of RSA public keys collected from unknown sources. If one could accurately link a particular key to its source, knowing the weakness of this source, he would be able to use specific factorization method to break the public key.

The aim of this thesis is to further extend the previous research of CROCS team \cite{svenda_1}\cite{svenda_3} lead by Petr Švenda. We presented the previous work in the chapter \ref{chapter-prev-work}. 

From them we obtained a dataset of more than 100 million public keys from 64 different sources. We designed a method to work with this dataset efficiently, perform an analysis and train and compare different machine learning methods while focusing on the extensibility of used classifiers. We implemented this method and used the application to compare different machine learning models. Namely, we focused on \textit{scikit} classifiers and neural networks implemented with \textit{TensorFlow}. 

Methodologically, we trained thousands of neural network models over several generated datasets with different hyperparameters. We compared and discussed the optimal usage of topology and hyperparameters. The result showed, that the classical methods had slightly lower overall accuracy when compared to neural networks. 

During the analysis, we discovered a slight bias in the most significant bits of a big number of sources from popular libraries from Microsoft or OpenSSL. In the end, we presented the optimal classifier to identify the used implementation of RSA.
